\documentclass[11pt]{article}
\usepackage{hyperref}
\usepackage{amsmath, amsfonts, amssymb, mathrsfs}
\usepackage{dcolumn}
\usepackage{caption}
\usepackage{subcaption}
\usepackage{filemod}
\usepackage{natbib}
\usepackage[ruled, vlined]{algorithm2e}
\usepackage{floatrow}
\usepackage{setspace}
\usepackage{verbatim}

\usepackage{graphicx}

\oddsidemargin=0.25in
\evensidemargin=0.25in
\textwidth=7in
\textheight=8.75in
\topmargin=-.5in
\addtolength{\oddsidemargin}{-.5in}
\addtolength{\evensidemargin}{-.5in}
\footskip=0.5in

\title{Computer Model Calibration for Large-Scale, Spatially Distributed Counts}
\author{Steven D. Barnett\thanks{Corresponding author: Department of Statistics, 
  Virginia Tech, {\tt sdbarnett@vt.edu}} \and Robert B. Gramacy\thanks{Department 
  of Statistics, Virginia Tech} \and
  Lauren J. Beesley\thanks{Statistical Sciences Group, Los Alamos National Laboratory} \and
  Dave Osthus\footnotemark[3]}
\date{\today}

\begin{document}

\maketitle
\bigskip

\begin{abstract} 
\textbf{Outline for the paper:} Motivate the need for computer model calibration
in the context of non-Gaussian data. Motivate the need to use an approximation
in computer model calibration of large-scale data (Vecchia). Introduce the IBEX
sky map as an excellent use case of both these barriers. Expand upon the
Kennedy \& O'Hagan canonical calibration framework. Propose using scaled Vecchia
as a GP surrogate. Illustrate the performance on simulated examples. Talk about
the need for adding discrepancy and propose model for doing so. Show the ability
to obtain good estimates of calibration parameters and out-of-sample predictions
with discrepancy analysis.
\end{abstract}

\noindent \textbf{Keywords:} vecchia, poisson, heliospheric science, gaussian process surrogate 

%\vfill

%%%%%%%%%%%%%%%%%%%%%%%%%%%%%%%%%%%%%%%%%%%%%%%%%%%%%%%%%%%%%%%%%%%%%%%%%%%%%%%
\section{Introduction}

\begin{itemize}
  \item Introduce the heliosphere, energetic neutral atoms, globally distributed flux, ribbon
  \item Introduce the Interstellar Boundary Explorer
  \item Motivate the use of computer models to understand the process that generates GDF/ribbon
  \item Introduce computer model simulations (ribbon, GDF)
  \item Explain current process of theory evaluation (Theseus, ribbon separation)
  \item Summary of work done on computer model calibration
  \item Summary of work modeling Poisson data in the context of computer model calibration
  \item Description of what we'll do: build a framework that's an extension of
    canonical Kenndy \& O'Hagan framework, use of scaled vecchia for handling large data,
    illustrate performance on simulated data, propose a method for handling discrepancy,
    demonstrate performance on real IBEX data.
\end{itemize}

%%%%%%%%%%%%%%%%%%%%%%%%%%%%%%%%%%%%%%%%%%%%%%%%%%%%%%%%%%%%%%%%%%%%%%%%%%%%%%%

%%%%%%%%%%%%%%%%%%%%%%%%%%%%%%%%%%%%%%%%%%%%%%%%%%%%%%%%%%%%%%%%%%%%%%%%%%%%%%%
\section{Review} 
\label{sec:review}

\begin{itemize}
  \item Review Gaussian processes and their use in surrogate modeling
  \item Review Kennedy O'Hagan framework
  \item Review Vecchia approximation and Scaled vecchia approximation
  \item Do we need to review any method that we'll later use for discrepancy?
\end{itemize}

%%%%%%%%%%%%%%%%%%%%%%%%%%%%%%%%%%%%%%%%%%%%%%%%%%%%%%%%%%%%%%%%%%%%%%%%%%%%%%%

%%%%%%%%%%%%%%%%%%%%%%%%%%%%%%%%%%%%%%%%%%%%%%%%%%%%%%%%%%%%%%%%%%%%%%%%%%%%%%%
\section{Model}
\label{sec:modeling}

\begin{itemize}
  \item Rehash the need to update KOH to account for Poisson data
  \item Show diagram of updated framework (McMC with Poisson likelihood)
  \item Explain the use of a no-bias set up to initially prove the capability
  \item Briefly show the math behind this
\end{itemize}

\subsection{Scaled Vecchia Gaussian Process Surrogate}

\subsection{Discrepancy Analysis}

\subsection{Estimation of calibration parameters}

%%%%%%%%%%%%%%%%%%%%%%%%%%%%%%%%%%%%%%%%%%%%%%%%%%%%%%%%%%%%%%%%%%%%%%%%%%%%%%%

%%%%%%%%%%%%%%%%%%%%%%%%%%%%%%%%%%%%%%%%%%%%%%%%%%%%%%%%%%%%%%%%%%%%%%%%%%%%%%%
\section{Simulated example: IBEX Skymap}
\label{sec:dgpal}

\begin{itemize}
  \item Go in depth on development of GDF and ribbon computer models
  \item Explain the parameters that govern the computer model output
  \item Run tests on noisy maps generated from known simulations
  \item Display distributions on calibration parameters that show the method
    is able to recover the truth
  \item Show prediction results on hold out data (RMSE? Score? Visual?)
\end{itemize}

%%%%%%%%%%%%%%%%%%%%%%%%%%%%%%%%%%%%%%%%%%%%%%%%%%%%%%%%%%%%%%%%%%%%%%%%%%%%%%%

%%%%%%%%%%%%%%%%%%%%%%%%%%%%%%%%%%%%%%%%%%%%%%%%%%%%%%%%%%%%%%%%%%%%%%%%%%%%%%%
\section{Discrepancy analysis in IBEX real data} 
\label{sec:experiments}

\begin{itemize}
  \item Illustrate the gap between computer model output and real data
  \item Recall how we will account for discrepancy (PCA?, GP?)
  \item Perform tests on examples where we know what the bias is. How well
    does our method account for this?
  \item Lack of competitors? (could we compare to calibration of
    ribbon-separated maps)
  \item Run our code on IBEX real data.
  \item We could display predictions for different values of the calibration
    parameters. Not sure if this is of interest.
  \item Display distributions on calibration parameters.
\end{itemize}

%%%%%%%%%%%%%%%%%%%%%%%%%%%%%%%%%%%%%%%%%%%%%%%%%%%%%%%%%%%%%%%%%%%%%%%%%%%%%%%

%%%%%%%%%%%%%%%%%%%%%%%%%%%%%%%%%%%%%%%%%%%%%%%%%%%%%%%%%%%%%%%%%%%%%%%%%%%%%%%
\section{Discussion} 
\label{sec:conclude}

\begin{itemize}
  \item Recap what we are proposing in the paper
  \item Highlight performance of calibration framework in prediction and
    estimation of calibration parameters
  \item Point out need to collaborate with theorists to improve computer
    model and get better estimates of calibration parameters
  \item Discuss a need to build out a more exhaustive procedure to perform
    hypothesis testing on which theoretical model is most probable to
    explain the data
  \item Acknowledge limitations (lack of fulll UQ with SVecchia estimates)
  \item Propose a fully Bayesian framework (teasing use of deep GPs)
\end{itemize}

\end{document}
