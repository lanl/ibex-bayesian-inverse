\documentclass[12pt]{article}
\usepackage{hyperref}
\usepackage{amsmath, amsfonts, amssymb, mathrsfs}
\usepackage{dcolumn}
\usepackage{caption}
\usepackage{subcaption}
\usepackage{filemod}
\usepackage{natbib}
\usepackage[ruled, vlined]{algorithm2e}
\usepackage{floatrow}
\usepackage{setspace}
\usepackage{verbatim}

\usepackage{graphicx}

\oddsidemargin=0.25in
\evensidemargin=0.25in
\textwidth=7in
\textheight=8.75in
\topmargin=-.5in
\addtolength{\oddsidemargin}{-.5in}
\addtolength{\evensidemargin}{-.5in}
\footskip=0.5in

\begin{document}

\title{Computer Model Calibration for Spatially Distributed
Counts} \author{Steven D. Barnett\thanks{Corresponding author
\href{mailto:sdbarnett@vt.edu}{\tt sdbarnett@vt.edu}, Department of Statistics,
Virginia Tech} \and Robert B. Gramacy\thanks{Department of Statistics, Virginia
Tech} \and Lauren J. Beesley\thanks{Statistical Sciences Group, Los Alamos
National Laboratory} \and Dave Osthus\footnotemark[3] \and Yifan Huang\thanks{
Nuclear and Particle Physics, AstroPhysics and Cosmology, Los Alamos National
Laboratory} \and Fan Guo\footnotemark[4] \and Eric J. Zirnstein\thanks{
Department of Astrophysical Sciences, Princeton University} \and Daniel B.
Reisenfeld\thanks{Space Science and Applications Group, Los Alamos National
Laboratory}}

\date{\today}

\maketitle

\vspace{-0.5cm}

\begin{abstract}
The Interstellar Boundary Explorer (IBEX) satellite detects energetic neutral
atoms (ENAs) and determines the rate at which they are generated in the
heliosphere. Computer models attempt to represent the physical process that
produces heliospheric ENAs through specified model parameters. Computer model
calibration enables statisticians to use data collected in a field experiment
along with simulator output from a variety of parameter settings to make
out-of-sample predictions and learn the posterior distributions of model
parameters. However, calibration typically assumes a Gaussian or continuous
response for both field and computer model data. We introduce a novel Markov
chain Monte Carlo calibration framework that accommodates the spatially
distributed Poisson counts collected by the IBEX satellite. Additionally, the
computer simulation in our application is limited in the amount of runs it can
perform, but each run's output is quite large. Therefore, we propose the use of
the Scaled Vecchia approximation as a Gaussian process (GP) surrogate. We
demonstrate the capability of our proposed framework through multiple simulated
examples and show that we can consistently recover the true parameters in a
discrepancy-free environment and obtain accurate out-of-sample prediction. We
apply this to the IBEX satellite data and the corresponding computer model
output.

\bigskip
\noindent \textbf{Key words:} computer experiments, simulator, Gaussian process
 surrogate, Poisson response, Vecchia approximation, Bayesian inference,
 heliospheric science, IBEX
\end{abstract}

\doublespacing % no double spacing for arXiv

%%%%%%%%%%%%%%%%%%%%%%%%%%%%%%%%%%%%%%%%%%%%%%%%%%%%%%%%%%%%%%%%%%%%%%%%%%%%%%%
\section{Introduction}
\label{sec:intro}

Solar wind, continously emitted from the Sun, is primarily made up of hydrogen
ions, or lone protons. These ions travel from the Sun, making their way through
the solar system, and out into interstellar space, forming what astrophysicists
call the heliosphere. Essentially, the heliosphere is an egg-shaped bubble
formed from the solar wind that acts as a barrier between our solar system and
interstellar space. Before the ions reach the boundary of our solar system,
they encounter the termination shock, where they slow down, absorb significant
amounts of energy, and become highly energetic, charged atoms. While in the
heliopause (the outer edge of the heliosphere), the ions can interact with
other particles in the interstellar medium. Sometimes these encounters result
in electron exchange, converting the charged atoms into what heliospheric
physicists call energetic neutral atoms (ENAs). Without charge, these ENAs
travel in a straight line, unaffected by magnetic fields. Some ENAs cross back
into the solar system and, depending on their path, can be detected on Earth.
Determining the rate at which ENAs are generated in different parts of the
heliosphere is key to understanding the makeup of this region at the boundary
of our solar system and interstellar space.

In 2008, the National Aeronautics and Space Administration (NASA) launched the
Interstellar Boundary Explorer (IBEX) into Earth's orbit as part of their Small
Explorer program. It's purpose is to detect ENAs coming from the heliopause,
record the approximate location of origin, thereby mapping out the generation
rate throughout the heliosphere. Specifically, the IBEX satellite contains an
instrument called the IBEX-Hi ENA imager \citep{funsten2009IBEXHiENA}, which
includes a series of plates that the ENAs travel through to be detected. The
IBEX satellite rotates on an axis, and over a period of six months, is able to
point at every part of the sky for a certain amount of time. Data collected
include the number of ENAs counted, the length of time the device was pointed
at a point in the sky, an estimated rate at which background ENAs (those not
originating from the heliopause), and the elliptical longitude and latitude for
this measurement.

Now, the heliosphere will be a foreign concept to anyone outside of the space
science field. But for our purposes, all we need is a simple analogy. We can
think of the heliosphere as a boat moving through water. Here the water
represents the interstellar medium. As a boat moves through water, the water is
disturbed. However, the turbulence created is not equal everywhere the boat is
touching the water. Turbulence is greater at the bow (front) and stern (back)
than on the sides (port and starboard) of the ship. Prior to the launch of the
IBEX satellite, space scientists had the same expectation for the heliosphere.
Higher rates of ENAs being generated would exist at the stern (space scientists
call this the nose) and bow (tail) of the helisophere due to its movement
through interstellar space. And that was found to be true. These higher rates
are referred to as globally distributed flux (GDF). But in a completely
unexpected finding \citep{mccomas2009aIBEX}, IBEX recorded a string of higher
rates of ENAs \citep{fuselier2009IBEXribbon}, or ribbon, being generated that
curved around the heliosphere. For the past decade and a half, space scientists
have proposed theories to explain the existence of the ribbon and the physical
process that generates it. Along with these proposed explanations, theorists
have developed computer simulations that produce maps of the heliosphere based
on the values of certain model parameters.

% Introduce computer model simulations (ribbon, GDF).
Many computer models to generate sky maps exist (???). In this paper we focus on two. One simulation to create variations of the GDF and one that offers and explanation of how the ribbon is produced.

Statistical computer model calibration utilizes both data collected from a
physical experiment and data generated from a computer simulation
representative of the physical process of interest. Often, the physical
experiment is either too expensive to run a sufficient number of times or it's
unethical to conduct. In this way, a computer model representing the physical
process allows a practitioner to learn more about the physical process without
the prohibitive expense. Use of this additional source of data (even with it's
potential bias, as no simulation is perfect) has been shown to improve
prediction and uncertainty quantification out-of-sample (ref?). It is often
impossible to control the values of certain variables in a physical experiment
(e.g. gravity), but that barrier does not exist in a computer simulation. These
calibration or tuning parameters can be of high scientific interest. Varying
them helps experimenters explore more of the input space, understand the
distribution of these parameters, and perhaps make inferential claims about
them. Parameters that are part of computer simulations developed by space
scientists to explain the existence of the GDF and ribbon fall in this
category.

%%% No calibration going on in IBEX (super novel)
Scientific investigation of the GDF and ribbon currently involve a long and
complex process. The noisy and irregular nature of the data has made it hard
for space scientists to perform analysis, especially separating signal from the
noise. The IBEX Science Operations Center (ISOC) perform some aggregation of
the raw data to produces maps, but these still retain a significant amount of
noise. To remedy this problem \citet{osthustheseus2023} propose a method called
Theseus, which takes the raw data and generates a much smoother representation
of the surface, using a series of Generative Additive Models (GAM) and
Projection Pursuit Regression. Once those Theseus maps are created, they are
separated into GDF-only and ribbon-only maps \citep{beesleyribbonsep2023}. With
the source of ENAs separated into two distinct maps, space scientists perform
analysis and conduct research exclusively for either the GDF or ribbon with the
respective map. Both Theseus and ribbon separation rely on many assumptions about the data and thus depart from the data generation process itself.

% Introduce a charicature of the IBEX problem

%%% SEPIA: Dave Higdon's LANL - we're going back to what Dave wanted to do, but couldn't at the time with
%%% computational limits
Summary of work done on computer model calibration. KOH. Reference Dave's work
on calibration. Wanted to do it, but had to do PCA stuff do to computational
limits.

Summary of work modeling Poisson data in the context of computer model
calibration. Mention that Katzfuss and Lawrence suggested use of SVecchia in
calibration

Description of what we'll do: build a framework that's an extension of
canonical Kenndy \& O'Hagan framework, use of scaled vecchia for handling large
data, illustrate performance on simulated data, demonstrate performance on real
IBEX data.

%%%%%%%%%%%%%%%%%%%%%%%%%%%%%%%%%%%%%%%%%%%%%%%%%%%%%%%%%%%%%%%%%%%%%%%%%%%%%%%

%%%%%%%%%%%%%%%%%%%%%%%%%%%%%%%%%%%%%%%%%%%%%%%%%%%%%%%%%%%%%%%%%%%%%%%%%%%%%%%
\section{Review}
\label{sec:review}

%%% novelty is SEPIA approach and Poisson
%%% get more into the detail of limitations that SEPIA and Vecchia help

\begin{itemize}
  \item Review Gaussian processes and their use in surrogate modeling
  \item Review Kennedy O'Hagan framework
  \item Review Vecchia approximation and Scaled vecchia approximation
  \item Do we need to review any method that we'll later use for discrepancy?
\end{itemize}

%%%%%%%%%%%%%%%%%%%%%%%%%%%%%%%%%%%%%%%%%%%%%%%%%%%%%%%%%%%%%%%%%%%%%%%%%%%%%%%

%%%%%%%%%%%%%%%%%%%%%%%%%%%%%%%%%%%%%%%%%%%%%%%%%%%%%%%%%%%%%%%%%%%%%%%%%%%%%%%
\section{Computer Model Calibration for Poisson-distributed Response}
\label{sec:poisson_calib}


%%% Vecchia should go in review (off the shelf)
%%% non-discrepancy version is pretty simple
%%% discrepancy free. illustrate it
%%% show posterior distributions

%%% subsection on last summer's work
%%% add discrepancy in section 4
\begin{itemize}
  \item Rehash the need to update KOH to account for Poisson data
  \item Show diagram of updated framework (McMC with Poisson likelihood)
  \item Explain the use of a no-bias set up to initially prove the capability
  \item Briefly show the math behind this
\end{itemize}

\subsection{Scaled Vecchia Gaussian Process Surrogate}

\subsection{Discrepancy Analysis}

\subsection{Estimation of calibration parameters}

%%%%%%%%%%%%%%%%%%%%%%%%%%%%%%%%%%%%%%%%%%%%%%%%%%%%%%%%%%%%%%%%%%%%%%%%%%%%%%%

%%%%%%%%%%%%%%%%%%%%%%%%%%%%%%%%%%%%%%%%%%%%%%%%%%%%%%%%%%%%%%%%%%%%%%%%%%%%%%%
\section{Scaled Vecchia}
\label{sec:poisson_calib}

%%% non-discrepancy version is pretty simple
%%% discrepancy free. illustrate it
%%% show posterior distributions

%%% subsection on last summer's work
%%% add discrepancy in section 4
\begin{itemize}
  \item Rehash the need to update KOH to account for Poisson data
  \item Show diagram of updated framework (McMC with Poisson likelihood)
  \item Explain the use of a no-bias set up to initially prove the capability
  \item Briefly show the math behind this
\end{itemize}

\subsection{Scaled Vecchia Gaussian Process Surrogate}

\subsection{Discrepancy Analysis}

\subsection{Estimation of calibration parameters}

%%%%%%%%%%%%%%%%%%%%%%%%%%%%%%%%%%%%%%%%%%%%%%%%%%%%%%%%%%%%%%%%%%%%%%%%%%%%%%%


%%%%%%%%%%%%%%%%%%%%%%%%%%%%%%%%%%%%%%%%%%%%%%%%%%%%%%%%%%%%%%%%%%%%%%%%%%%%%%%
\section{Simulated example: IBEX Skymap}
\label{sec:dgpal}

\begin{itemize}
  \item Go in depth on development of GDF and ribbon computer models
  \item Explain the parameters that govern the computer model output
  \item Run tests on noisy maps generated from known simulations
  \item Display distributions on calibration parameters that show the method
    is able to recover the truth
  \item Show prediction results on hold out data (RMSE? Score? Visual?)
\end{itemize}

%%%%%%%%%%%%%%%%%%%%%%%%%%%%%%%%%%%%%%%%%%%%%%%%%%%%%%%%%%%%%%%%%%%%%%%%%%%%%%%

%%%%%%%%%%%%%%%%%%%%%%%%%%%%%%%%%%%%%%%%%%%%%%%%%%%%%%%%%%%%%%%%%%%%%%%%%%%%%%%
\section{Discrepancy analysis in IBEX real data}
\label{sec:experiments}

\begin{itemize}
  \item Illustrate the gap between computer model output and real data
  \item Recall how we will account for discrepancy (PCA?, GP?)
  \item Perform tests on examples where we know what the bias is. How well
    does our method account for this?
  \item Lack of competitors? (could we compare to calibration of
    ribbon-separated maps)
  \item Run our code on IBEX real data.
  \item We could display predictions for different values of the calibration
    parameters. Not sure if this is of interest.
  \item Display distributions on calibration parameters.
\end{itemize}

%%%%%%%%%%%%%%%%%%%%%%%%%%%%%%%%%%%%%%%%%%%%%%%%%%%%%%%%%%%%%%%%%%%%%%%%%%%%%%%

%%%%%%%%%%%%%%%%%%%%%%%%%%%%%%%%%%%%%%%%%%%%%%%%%%%%%%%%%%%%%%%%%%%%%%%%%%%%%%%
\section{Discussion}
\label{sec:discuss}

\begin{itemize}
  \item Recap what we are proposing in the paper
  \item Highlight performance of calibration framework in prediction and
    estimation of calibration parameters
  \item Point out need to collaborate with theorists to improve computer
    model and get better estimates of calibration parameters
  \item Discuss a need to build out a more exhaustive procedure to perform
    hypothesis testing on which theoretical model is most probable to
    explain the data
  \item Acknowledge limitations (lack of fulll UQ with SVecchia estimates)
  \item Propose a fully Bayesian framework (teasing use of deep GPs)
\end{itemize}

\subsection*{Acknowledgments}

RBG and SDB are grateful for funding from NSF CMMI 2152679. This work has been
approved for public release under LA-UR-??-?????. SDB, DO and LJB were funded
by Laboratory Directed Research and Development (LDRD) Project 20220107DR.

\bibliography{poisson_calib}
\bibliographystyle{jasa}

\appendix

\end{document}
