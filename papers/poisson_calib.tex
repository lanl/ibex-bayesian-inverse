\documentclass[12pt]{article}
\usepackage{hyperref}
\usepackage{amsmath, amsfonts, amssymb, mathrsfs}
\usepackage{dcolumn}
\usepackage{caption}
\usepackage{subcaption}
\usepackage{filemod}
\usepackage{natbib}
\usepackage[ruled, vlined]{algorithm2e}
\usepackage{floatrow}
\usepackage{setspace}
\usepackage{verbatim}

\usepackage{graphicx}

\oddsidemargin=0.25in
\evensidemargin=0.25in
\textwidth=7in
\textheight=8.75in
\topmargin=-.5in
\addtolength{\oddsidemargin}{-.5in}
\addtolength{\evensidemargin}{-.5in}
\footskip=0.5in

\begin{document}

\title{Computer Model Calibration for Large-Scale, Spatially Distributed Counts}
\author{Steven D. Barnett\thanks{Corresponding author \href{mailto:sdbarnett@vt.edu}{\tt sdbarnett@vt.edu},
  Department of Statistics, Virginia Tech} \and Robert B. Gramacy\thanks{Department
  of Statistics, Virginia Tech} \and
  Lauren J. Beesley\thanks{Statistical Sciences Group, Los Alamos National Laboratory} \and
  Dave Osthus\footnotemark[3] \and Yifan Huang\thanks{Nuclear and Particle Physics,
  AstroPhysics and Cosmology, Los Alamos National Laboratory} \and Fan Guo\footnotemark[4]
  \and Eric J. Zirnstein\thanks{Department of Astrophysical Sciences, Princeton University}
  \and Daniel B. Reisenfeld\thanks{Space Science and Applications Group, Los Alamos National
  Laboratory}}
\date{\today}

\maketitle

\vspace{-0.5cm}

\begin{abstract}
The Interstellar Boundary Explorer (IBEX) satellite detects energetic neutral
atoms (ENAs) and determines the rate at which they are generated in the heliosphere.
Computer models attempt to represent the physical process that produces these ENAs
based on varying the value of certain parameters. Computer model calibration enables
scientists to use data collected in a field experiment along with simulator output
to predict out-of-sample and learn the distribution of parameters. We introduce an
extension of the Kennedy \& O'Hagan canonical calibration framework that handles
Poisson-distributed data collected by the IBEX satellite. The resolution at which
data is collected prohibits us from using typical models. We propose the use of
the Scaled Vecchia approximation as a Gaussian process (GP) surrogate. We demonstrate
the capability of our proposed framework through multiple simulated examples.
Additionally, we show that we can recover the true parameters in a discrepancy-free
environment. We provide guidance on how to account for discrepancy and show that
we can obtain accurate out of sample prediction.

\noindent \textbf{Key words:} simulator, Gaussian process surrogate, Poisson response,
 Vecchia approximation, Bayesian inference, discrepancy analysis, heliospheric
 science, IBEX
\end{abstract}

\doublespacing

%%%%%%%%%%%%%%%%%%%%%%%%%%%%%%%%%%%%%%%%%%%%%%%%%%%%%%%%%%%%%%%%%%%%%%%%%%%%%%%
\section{Introduction}
\label{sec:intro}

%%% No calibration going on in IBEX (super novel)
%%% like the narrative of no bias, then bias
%%% SEPIA: Dave Higdon's LANL - we're going back to what Dave wanted to do, but couldn't at the time with
%%% computational limits

\begin{itemize}
  \item Introduce the heliosphere, energetic neutral atoms, globally distributed flux, ribbon
  \item Introduce the Interstellar Boundary Explorer
  \item Motivate the use of computer models to understand the process that generates GDF/ribbon
  \item Introduce computer model simulations (ribbon, GDF)
  \item Explain current process of theory evaluation (Theseus, ribbon separation)
  \item Summary of work done on computer model calibration
  \item Summary of work modeling Poisson data in the context of computer model calibration
  \item Description of what we'll do: build a framework that's an extension of
    canonical Kenndy \& O'Hagan framework, use of scaled vecchia for handling large data,
    illustrate performance on simulated data, propose a method for handling discrepancy,
    demonstrate performance on real IBEX data.
\end{itemize}

%%%%%%%%%%%%%%%%%%%%%%%%%%%%%%%%%%%%%%%%%%%%%%%%%%%%%%%%%%%%%%%%%%%%%%%%%%%%%%%

%%%%%%%%%%%%%%%%%%%%%%%%%%%%%%%%%%%%%%%%%%%%%%%%%%%%%%%%%%%%%%%%%%%%%%%%%%%%%%%
\section{Review}
\label{sec:review}

%%% Vecchia should go in review (off the shelf)
%%% novelty is SEPIA approach and Poisson
%%% get more into the detail of limitations that SEPIA and Vecchia help

\begin{itemize}
  \item Review Gaussian processes and their use in surrogate modeling
  \item Review Kennedy O'Hagan framework
  \item Review Vecchia approximation and Scaled vecchia approximation
  \item Do we need to review any method that we'll later use for discrepancy?
\end{itemize}

%%%%%%%%%%%%%%%%%%%%%%%%%%%%%%%%%%%%%%%%%%%%%%%%%%%%%%%%%%%%%%%%%%%%%%%%%%%%%%%

%%%%%%%%%%%%%%%%%%%%%%%%%%%%%%%%%%%%%%%%%%%%%%%%%%%%%%%%%%%%%%%%%%%%%%%%%%%%%%%
\section{Model}
\label{sec:modeling}

%%% non-discrepancy version is pretty simple
%%% discrepancy free. illustrate it
%%% show posterior distributions

%%% subsection on last summer's work
%%% add discrepancy in section 4
\begin{itemize}
  \item Rehash the need to update KOH to account for Poisson data
  \item Show diagram of updated framework (McMC with Poisson likelihood)
  \item Explain the use of a no-bias set up to initially prove the capability
  \item Briefly show the math behind this
\end{itemize}

\subsection{Scaled Vecchia Gaussian Process Surrogate}

\subsection{Discrepancy Analysis}

\subsection{Estimation of calibration parameters}

%%%%%%%%%%%%%%%%%%%%%%%%%%%%%%%%%%%%%%%%%%%%%%%%%%%%%%%%%%%%%%%%%%%%%%%%%%%%%%%

%%%%%%%%%%%%%%%%%%%%%%%%%%%%%%%%%%%%%%%%%%%%%%%%%%%%%%%%%%%%%%%%%%%%%%%%%%%%%%%
\section{Simulated example: IBEX Skymap}
\label{sec:dgpal}

\begin{itemize}
  \item Go in depth on development of GDF and ribbon computer models
  \item Explain the parameters that govern the computer model output
  \item Run tests on noisy maps generated from known simulations
  \item Display distributions on calibration parameters that show the method
    is able to recover the truth
  \item Show prediction results on hold out data (RMSE? Score? Visual?)
\end{itemize}

%%%%%%%%%%%%%%%%%%%%%%%%%%%%%%%%%%%%%%%%%%%%%%%%%%%%%%%%%%%%%%%%%%%%%%%%%%%%%%%

%%%%%%%%%%%%%%%%%%%%%%%%%%%%%%%%%%%%%%%%%%%%%%%%%%%%%%%%%%%%%%%%%%%%%%%%%%%%%%%
\section{Discrepancy analysis in IBEX real data}
\label{sec:experiments}

\begin{itemize}
  \item Illustrate the gap between computer model output and real data
  \item Recall how we will account for discrepancy (PCA?, GP?)
  \item Perform tests on examples where we know what the bias is. How well
    does our method account for this?
  \item Lack of competitors? (could we compare to calibration of
    ribbon-separated maps)
  \item Run our code on IBEX real data.
  \item We could display predictions for different values of the calibration
    parameters. Not sure if this is of interest.
  \item Display distributions on calibration parameters.
\end{itemize}

%%%%%%%%%%%%%%%%%%%%%%%%%%%%%%%%%%%%%%%%%%%%%%%%%%%%%%%%%%%%%%%%%%%%%%%%%%%%%%%

%%%%%%%%%%%%%%%%%%%%%%%%%%%%%%%%%%%%%%%%%%%%%%%%%%%%%%%%%%%%%%%%%%%%%%%%%%%%%%%
\section{Discussion}
\label{sec:discuss}

\begin{itemize}
  \item Recap what we are proposing in the paper
  \item Highlight performance of calibration framework in prediction and
    estimation of calibration parameters
  \item Point out need to collaborate with theorists to improve computer
    model and get better estimates of calibration parameters
  \item Discuss a need to build out a more exhaustive procedure to perform
    hypothesis testing on which theoretical model is most probable to
    explain the data
  \item Acknowledge limitations (lack of fulll UQ with SVecchia estimates)
  \item Propose a fully Bayesian framework (teasing use of deep GPs)
\end{itemize}

\subsection*{Acknowledgments}

RBG and SDB are grateful for funding from NSF CMMI 2152679. This work has been
approved for public release under LA-UR-??-?????. SDB, DO and LJB were funded
by Laboratory Directed Research and Development (LDRD) Project 20220107DR.

\bibliography{poisson_calib}
\bibliographystyle{jasa}

\appendix

\end{document}
