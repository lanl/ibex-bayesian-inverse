\documentclass[12pt]{article}
\usepackage[ruled, vlined]{algorithm2e}
\usepackage{amsmath, amsfonts, amssymb, mathrsfs}
\usepackage{caption}
\usepackage{dcolumn}
\usepackage{filemod}
\usepackage{floatrow}
\usepackage{gensymb}
\usepackage{graphicx}
\usepackage{hyperref}
\usepackage{natbib}
\usepackage{setspace}
\usepackage{subcaption}
\usepackage{verbatim}

\oddsidemargin=0.25in
\evensidemargin=0.25in
\textwidth=7in
\textheight=8.75in
\topmargin=-.5in
\addtolength{\oddsidemargin}{-.5in}
\addtolength{\evensidemargin}{-.5in}
\footskip=0.5in

\begin{document}

\title{TITLE} \author{Steven D. Barnett\thanks{Corresponding author
\href{mailto:sdbarnett@vt.edu}{\tt sdbarnett@vt.edu}, Department of Statistics,
Virginia Tech} \and Robert B. Gramacy\thanks{Department of Statistics, Virginia
Tech} \and Lauren J. Beesley\thanks{Statistical Sciences Group, Los Alamos
National Laboratory} \and Dave Osthus\footnotemark[3] \and Yifan Huang\thanks{
Nuclear and Particle Physics, AstroPhysics and Cosmology, Los Alamos National
Laboratory} \and Fan Guo\footnotemark[4] \and Eric J. Zirnstein\thanks{
Department of Astrophysical Sciences, Princeton University} \and Daniel B.
Reisenfeld\thanks{Space Science and Applications Group, Los Alamos National
Laboratory}}
\date{}

\maketitle

\vspace{-0.5cm}

\begin{abstract}
ABSTRACT.

\bigskip
\noindent \textbf{Key words:} KEY WORDS
\end{abstract}

\doublespacing % no double spacing for arXiv

%%%%%%%%%%%%%%%%%%%%%%%%%%%%%%%%%%%%%%%%%%%%%%%%%%%%%%%%%%%%%%%%%%%%%%%%%%%%%%%

\section{Introduction}
\label{sec:intro}

%% IBEX
%% - Talk about the satellite
%% - Explain the mission
%% - Introduce the idea of a computer model
%% - Explain desire to understand the distribution of parameters

%% PROBLEM
%% - Classic Inverse Problem
%% - Mathematical model is expensive (and perhaps not accurate)
%% - Functional output is limited, but in high dimension
%% - Physical observations are counts

%% OTHER WORK
%% - Bayesian Inverse Problems (BIP) w/ GP
%%   - Not large scale
%%   - Not Poisson
%% - BIP w/ functional data
%%   - more intuitive and scaleable way
%% - Computer Model Calibration
%%   - SEPIA / GPMSA
%%     - Not Poisson
%%     - Simpler Approach

%%%%%%%%%%%%%%%%%%%%%%%%%%%%%%%%%%%%%%%%%%%%%%%%%%%%%%%%%%%%%%%%%%%%%%%%%%%%%%%

%%%%%%%%%%%%%%%%%%%%%%%%%%%%%%%%%%%%%%%%%%%%%%%%%%%%%%%%%%%%%%%%%%%%%%%%%%%%%%%

% SECTION 2: Review:
\section{Review}
\label{sec:review}

%------------------------------------------------------------------------------
%% SECTION 2.1 - Bayesian Inverse Problems
\subsection{Bayesian Inverse Problems}
\label{sec:bayes_inv}
%------------------------------------------------------------------------------

%% BAYESIAN INVERSE PROBLEMS
%% - What is an inverse problem and why do we care?
%% - Obtained observations from a physical process to learn more about it
%% - Observations are noisy
%% - We have some knowledge of the workings of the true process
%% - We can build a mathematical model, and therefore a computer model
%% - Classic IP: evaluate model as much as possible to find parameters/inputs
%%   that produce output aligned with the observations
%% - Desire full UQ on parameters
%% - Go Bayesian
%% - Show Prior / Posterior
%% - No analytical form, must resort to MCMC
%% - LEAD IN: Computer model is expensive. Need a surrogate.

%------------------------------------------------------------------------------
%% SECTION 2.2 - Gaussian Process Surrogates:
\subsection{Gaussian Process Surrogates}
\label{sec:gp_surr}
%------------------------------------------------------------------------------

%% GAUSSIAN PROCESS SURROGATES
%% - What is the motivation? Expensive computer experiments
%% - Set of n observations that follow a MVN
%% - Often mean-zero. In practice this means subtracting off the mean
%% - Results in all the action being in the covariance function
%% - Covariance depends on pairwise distances
%% - Explain some different kernels
%% - Talk about prediction. Show Kriging equations
%% - Nonparametric, flexible regression tool
%% - A lot of different applications
%% - Become popular in computer experiments because of interpolation
%% - This is what we want
%% - Bottleneck is O(n^3)
%% - LEAD IN: Output can be large. Need an approximation.

%------------------------------------------------------------------------------
%% SECTION 2.3 - SEPIA / GPMSA
\subsection{SEPIA / GPMSA}
\label{sec:sepia}
%------------------------------------------------------------------------------

%% SEPIA / GPMSA
%% - O(n^3) bottleneck is nothing new
%% - Higdon et al. faced this at LANL in the early 2000s
%% - Their situation is unique in that n was small, but dimension is high
%% - Same situation as IBEX
%% - Instead of modeling the output as scalar, they did functional
%% - PCA
%%   - Show some equations
%%   - Maybe a graphic
%% - Must choose basis
%% - Used in calibration (an inverse problem), but does not support Poisson

%%%%%%%%%%%%%%%%%%%%%%%%%%%%%%%%%%%%%%%%%%%%%%%%%%%%%%%%%%%%%%%%%%%%%%%%%%%%%%%

%%%%%%%%%%%%%%%%%%%%%%%%%%%%%%%%%%%%%%%%%%%%%%%%%%%%%%%%%%%%%%%%%%%%%%%%%%%%%%%

% SECTION 3: Methods
\section{Methods}
\label{sec:methods}

%------------------------------------------------------------------------------
%% SECTION 3.1 - Generalized Bayesian Inverse Problems with GP Surrogates
\subsection{Generalized Bayesian Inverse Problems with GP Surrogates}
\label{sec:gen_bayes_inv}
%------------------------------------------------------------------------------

%% GENERALIZED BAYESIAN INVERSE PROBLEMS WITH GP SURROGATES
%% - Return to review, but with general response
%% - Use link function
%% - GPMSA / SEPIA cannot do this
%% - Show some equations

%------------------------------------------------------------------------------
%% SECTION 3.1.1 - Examples
\subsubsection{Examples}
\label{sec:gen_bayes_inv_ex}
%------------------------------------------------------------------------------

%% EXAMPLES
%% - 1D example with Poisson response
%%   - note that we are treating the response as a scalar
%%   - response is Poisson
%%   - Graphic showing field / computer model data, fit, and posterior over parameters
%% - 2D example with Poisson response
%%   - Same Thing as above
%%   - LEAD IN: capacity of GP is stretched

%------------------------------------------------------------------------------
%% SECTION 3.2 - Vecchia Approximation
\subsection{Vecchia Approximation}
\label{sec:vecchia}
%------------------------------------------------------------------------------

%% VECCHIA APPROXIMATION
%% - Review improvement in computation
%% - Higdon may have modeled it this way if he had the resources
%% - List GP approximations from past decade
%% - We use Vecchia
%% - Show formulas
%% - Specify that we use Scaled Vecchia approximation, what it is

%------------------------------------------------------------------------------
%% SECTION 3.2.1 - Examples
\subsubsection{Examples}
\label{sec:vecchia_ex}
%------------------------------------------------------------------------------

%% EXAMPLES
%% - 2D example with LOTs of data
%%   - Show results. Compare to previous
%% - Show timing test on simulated data
%%   - Graphic should include fitting and prediction
%%   - Include different number of bases for SEPIA, maybe compared with different m
%%   - Show performance of different models

%%%%%%%%%%%%%%%%%%%%%%%%%%%%%%%%%%%%%%%%%%%%%%%%%%%%%%%%%%%%%%%%%%%%%%%%%%%%%%%

%%%%%%%%%%%%%%%%%%%%%%%%%%%%%%%%%%%%%%%%%%%%%%%%%%%%%%%%%%%%%%%%%%%%%%%%%%%%%%%

% SECTION 4: Simulations
\section{Simulations}
\label{sec:sims}

%------------------------------------------------------------------------------
%% SECTION 4.1 - Heliospheric Science
\subsection{Heliospheric Science}
\label{sec:helio_science}
%------------------------------------------------------------------------------

%% HELIOSPHERIC SCIENCE
%% - Short review of heliosphere
%% - Behavior of Energetic Neutral Atoms

%% HELIOSPHERE COMPUTER MODEL
%% - Introduce computer model
%% - Explanation of parameters in computer model
%% - Introduce the data from the computer model

%------------------------------------------------------------------------------
%% SECTION 4.2 - Experiment
\subsection{Experiment}
\label{sec:ibex_sim_exp}
%------------------------------------------------------------------------------

%% EXPERIMENT
%% - Explain set up of the experiment
%% - Show results from simulated data
%% - Lead-in to real data

%%%%%%%%%%%%%%%%%%%%%%%%%%%%%%%%%%%%%%%%%%%%%%%%%%%%%%%%%%%%%%%%%%%%%%%%%%%%%%%

%%%%%%%%%%%%%%%%%%%%%%%%%%%%%%%%%%%%%%%%%%%%%%%%%%%%%%%%%%%%%%%%%%%%%%%%%%%%%%%

% SECTION 5: Real Data
\section{Real Data}
\label{sec:ibex_real}

%------------------------------------------------------------------------------
%% SECTION 5.1 - IBEX
\subsection{IBEX}
\label{sec:ibex}
%------------------------------------------------------------------------------

%% IBEX
%% - Reintroduce satellite
%% - Explain details of data
%% - Explain the years gathered

%------------------------------------------------------------------------------
%% SECTION 5.2 - EXPERIMENT
\subsection{Experiment}
\label{sec:ibex_exp}
%------------------------------------------------------------------------------

%% EXPERIMENT
%% - Explain set up of the experiment
%% - Show results from the real data
%% - Comment on the missalignment between simulator and reality

%------------------------------------------------------------------------------
%% SECTION 5.3 - DISCREPANCY
\subsection{Discrepancy}
\label{sec:ibex_discrep}
%------------------------------------------------------------------------------

%% DISCREPANCY
%% - Refer to KOH, Higdon, and their use of discrepancy
%% - Introduce simple scaling discrepancy (iniclude some equations)
%% - Refer to results in Appendix for simulated data where we discover the true scale
%% - Explain set up for IBEX calibration with discrepancy
%% - Show results
%% - Comment on complexity of discrepancy. Scale isn't sufficient.
%% - Perhaps a stationary GP is not even sufficient

%%%%%%%%%%%%%%%%%%%%%%%%%%%%%%%%%%%%%%%%%%%%%%%%%%%%%%%%%%%%%%%%%%%%%%%%%%%%%%%

%%%%%%%%%%%%%%%%%%%%%%%%%%%%%%%%%%%%%%%%%%%%%%%%%%%%%%%%%%%%%%%%%%%%%%%%%%%%%%%

% SECTION 6: Discussion
\section{Discussion}
\label{sec:discuss}

\subsection*{Acknowledgments}

RBG and SDB are grateful for funding from NSF CMMI 2152679. This work has been
approved for public release under LA-UR-??-?????. SDB, DO and LJB were funded
by Laboratory Directed Research and Development (LDRD) Project 20220107DR.

\bibliography{ibex_bayes_inv}
\bibliographystyle{jasa}

\appendix

\end{document}
