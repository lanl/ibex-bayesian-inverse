\documentclass[12pt]{article}
\usepackage[ruled, vlined]{algorithm2e}
\usepackage{amsmath, amsfonts, amssymb, mathrsfs}
\usepackage{caption}
\usepackage{dcolumn}
\usepackage{filemod}
\usepackage{floatrow}
\usepackage{gensymb}
\usepackage{graphicx}
\usepackage{hyperref}
\usepackage{natbib}
\usepackage{setspace}
\usepackage{subcaption}
\usepackage{verbatim}

\oddsidemargin=0.25in
\evensidemargin=0.25in
\textwidth=7in
\textheight=8.75in
\topmargin=-.5in
\addtolength{\oddsidemargin}{-.5in}
\addtolength{\evensidemargin}{-.5in}
\footskip=0.5in

\begin{document}

\pagenumbering{gobble}

\begin{abstract}

The computer model calibration literature is well-suited for noisy, Gaussian field
data and small-scale, smooth computer model output that can be modeled with a
Gaussian process (GP). However, not all scientific problems served by computer model
calibration fit into that mold. For example, we are working with count data collected
by a satellite detector. Counts are collected over a grid of longitudes and
latitudes, creating datasets in the range of tens or hundreds of thousands. Computer
model output simulating this process is also large, preventing the use of a
traditional GP surrogate. We introduce a novel Markov chain Monte Carlo calibration
framework that accommodates the spatially distributed Poisson counts collected by the
Interstellar Boundary Explorer (IBEX) satellite. Additionally, we propose the use of
the Scaled Vecchia approximation as a GP surrogate for the computer simulation. We
demonstrate the capability of our proposed framework through multiple simulated
examples and show that we can consistently recover the true parameters in a
discrepancy-free environment and obtain accurate out-of-sample prediction. We apply
this to the IBEX satellite data and the corresponding computer model output.

The solar wind, continously emitted from the Sun, is primarily made up of hydrogen
ions, or lone protons. These ions travel through the solar system and out into
interstellar space. As the ions reach the boundary of our solar system, they
encounter the termination shock, where they slow down, absorb significant amounts of
energy, and become highly energetic, charged atoms. These hydrogen ions form an
egg-shaped bubble encapsulating our solar system called the heliosphere, which acts
as a barrier between our solar system and interstellar space. While in this region
called the heliopause (the outer edge of the heliosphere), the ions can interact with
other particles in the interstellar medium. Sometimes these encounters result in
electron exchange, converting the charged atoms into what heliospheric physicists
call energetic neutral atoms (ENAs). Without charge, these ENAs travel in a straight
line, unaffected by magnetic fields. Some ENAs cross back into the solar system and,
depending on their path, can be detected in Earth's orbit. Determining the rate at
which ENAs are generated in different parts of the heliosphere is key to
understanding the makeup of this region at the boundary of our solar system and
interstellar space.

In this pursuit, the National Aeronautics and Space Administration (NASA) launched
the IBEX satellite into Earth's orbit in 2008 as part of their Small Explorer
program. IBEX continuously detects ENAs and records their energy level and
approximate location of origin, providing sufficient data to estimate the rate at
which ENAs are created throughout the heliopause. Specifically, the IBEX satellite
contains an instrument called the IBEX-Hi ENA imager, which includes a series of
plates that the ENAs travel through to be detected. The IBEX satellite rotates on an
axis, and over a period of six months, is able to point at every part of the sky for
some interval, or exposure time. This spatial map of ENA rates (referred to by space
scientists as a \textit{sky map}) is used to analyze, make conjectures about, and
ultimately develop theories for the heliosphere, its many properties, and the
processes that govern its creation.

The heliosphere is like a boat moving through water. Here the water represents
interstellar medium. As a boat moves through water, the water is disturbed. However,
the turbulence created is not equal at each location where the boat is in contact
with the water. Turbulence is greater at the front (bow) and back (stern) than on the
sides (port and starboard) of the ship. Prior to the launch of the IBEX satellite,
space scientists had the same expectation for the heliosphere. Higher rates of ENAs
being generated would exist at the stern (space scientists call this the
\textit{nose}) and bow (\textit{tail}) of the helisophere due to its movement through
interstellar space. Data observed found this to be true. These higher rates are
referred to as \textit{globally distributed flux (GDF)}. But in a completely
unexpected finding , IBEX also recorded a string of higher rates of ENAs being
generated that curved around the heliosphere. Scientists now refer to this phenomenom
as the \textit{ribbon}. For the past fifteen years, space scientists have conducted
extensive research and proposed dozens of theories to explain the existence of the
ribbon and the physical process that generates it. This has led to development of
computer models to simulate the creation of sky maps. These computer simulations rely
on a number of parameters that can be varied, thus modifying the shape and intensity
of both ribbon and GDF. So far, minimal work has been done to validate theories or
further understand the heliosphere by pairing the computer model output with real
data collected by IBEX.

Statistical computer model calibration utilizes both data collected from a physical
experiment and data generated from a computer simulation representative of the
physical process of interest. Often, the physical experiment is either too expensive
to run a sufficient number of times (e.g. launching hundreds of IBEX satellites) or
it is unethical to conduct. In this way, a computer model representing the physical
process allows a practitioner to learn more about the physical process without the
prohibitive expense. Use of this additional source of data (even with its potential
bias) has been shown to improve prediction and uncertainty quantification
out-of-sample. For example, it is often impossible to control the values of certain
variables in a physical experiment (e.g. gravity), but that barrier does not exist in
a computer simulation. Parameters in computer simulations developed by space
scientists to explain the existence of the GDF and ribbon fall in this category.

Computer experiments do not face the same ethical and practical obstacles of physical
experiments. Although some simulations are very quick to evaluate, frequently they
can be computationally expensive. In our case, an individual run of the IBEX computer
model can take up to 16 hours, limiting the supply of computer model data. In such
situations, a GP surrogate is often fit to provide fast predictions out of sample.
Adding another layer of complexity, the output from the IBEX simulation is not only
expensive to obtain, but is extremely high dimensional. Each run of the simulation
produces a sky map represented by a vector of 16,200 ENA rates. GPs struggle with
such big data because they must store and decompose matrices that grow quadratically
with training data size. Previous work has attempted to mitigate this with a basis
representation via principal components to reduce the dimension of the data. However,
much has changed recently in the landscape of GP approximation. We propose a modern
approach using the Scaled Vecchia approximation. Therefore, our contributions are
three-fold: 1) Introduce an MCMC computer model calibration framework for any type of
physical response in high dimensions (we focus on Poisson-distributed data), 2) Allow
for processing large-scale, high-dimensional computer model output in a reasonable
time frame, and 3) For the first time, illustrate how IBEX data in its raw form can
be used to make inferential claims about the heliosphere.

\end{abstract}

\end{document}
